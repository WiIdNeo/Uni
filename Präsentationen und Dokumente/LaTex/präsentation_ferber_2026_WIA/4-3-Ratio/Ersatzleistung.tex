\documentclass[aspectratio=43]{beamer}

% ---------------------------------------------------------
% PAKETE
% ---------------------------------------------------------
\usepackage{xcolor}
\usepackage{graphicx}

% ---------------------------------------------------------
% FARBEN
% ---------------------------------------------------------
\definecolor{mygreen}{RGB}{14,127,38}
\definecolor{lightgraybg}{RGB}{217,217,217}

% ---------------------------------------------------------
% BEAMER-GRUNDLAGEN
% ---------------------------------------------------------
\setbeamertemplate{navigation symbols}{}
\setbeamersize{text margin left=0cm, text margin right=0cm}

\setbeamercolor{headlinebox}{bg=white, fg=black}
\setbeamercolor{footlinebox}{bg=white, fg=black}
\setbeamercolor{frametitle}{fg=mygreen}
\setbeamercolor{normal text}{fg=black}

% ---------------------------------------------------------
% SEMANTISCHES ÜBERTHEMA (VARIABEL)
% ---------------------------------------------------------
\newcommand{\overheader}{}

\newcommand{\setoverheader}[1]{%
  \renewcommand{\overheader}{#1}
}

% ---------------------------------------------------------
% KOPFZEILE
% ---------------------------------------------------------
\setbeamertemplate{headline}{
  \begin{beamercolorbox}[wd=\paperwidth,ht=1.2cm,dp=0cm]{headlinebox}
    \vspace{0.2cm}
    \hbox to \paperwidth{
      \hspace{0.4cm}
      {\color{mygreen}\Large\textbf\overheader}
      \hfill
      \raisebox{-0.3cm}{\includegraphics[height=1cm]{logo.png}}
      \hspace{0.4cm}
    }
    \vspace{-0.1cm}

    {\color{mygreen}\rule{\paperwidth}{2pt}}
  \end{beamercolorbox}
}

% ---------------------------------------------------------
% FUSSZEILE
% ---------------------------------------------------------
\setbeamertemplate{footline}{
  \begin{beamercolorbox}[wd=\paperwidth,ht=0.6cm,dp=0cm]{footlinebox}
    \vbox to 0.6cm{
      \vspace{0.1cm}
      \hbox to \paperwidth{
        \hspace{0.3cm}\insertshorttitle
        \hfill
        \insertframenumber\hspace{0.3cm}
      }
      \vfill
    }
  \end{beamercolorbox}
}

% ---------------------------------------------------------
% HINTERGRUND
% ---------------------------------------------------------
\setbeamertemplate{background}{
  {\color{lightgraybg}\rule{\paperwidth}{\paperheight}}
}

% ---------------------------------------------------------
% TITELINFOS
% ---------------------------------------------------------
\title[Vorstellung]{Vorstellung: Firma, Bereich und Tätigkeit}
\author{Colin Hanschmann}
\date{\today}




\begin{document}

% ---------------------------------------------------------
% ERSTE FOLIE: NUR BILD
% ---------------------------------------------------------
{
\setbeamertemplate{headline}{}
\setbeamertemplate{footline}{}
\setbeamertemplate{background}{}

\begin{frame}[plain]
  \makebox[0pt][l]{%
    \includegraphics[width=\paperwidth,height=\paperheight]{top_frame.png}%
  }
\end{frame}
}

% ---------------------------------------------------------
% TITELFOLIE
% ---------------------------------------------------------
\setoverheader{}

\begin{frame}
  \titlepage
\end{frame}

% ---------------------------------------------------------
% UNTERNEHMEN
% ---------------------------------------------------------
\section{Unternehmen}
\setoverheader{G.E.O.S. - Allgemein}

\begin{frame}
\frametitle{Überblick}

\begin{itemize}
  \item International tätiges Ingenieurunternehmen \\mit rund 150 Jahren Geschichte
  \item Geschäftsfelder: 
  \begin{itemize}
    \item Geotechnik
    \item Rohstoffwirtschaft
    \item Bergbausanierung
    \item Infrastruktur
    \item Genehmigungsmanagement
    \item Hydrogeologie
    \item Geoinformatik
    \item Strahlenschutz
  \end{itemize}
\end{itemize}
\end{frame}

\begin{frame}
\frametitle{Kennzahlen}

\begin{itemize}
  \item 225 Spezialisten
  \item 8 Standorte 
  \item über 1.300 Kunden
  \item rund 500 Projekte pro Jahr
\end{itemize}
\end{frame}

% ---------------------------------------------------------
% WISUTEC
% ---------------------------------------------------------
\section{Standort}
\setoverheader{Standort Chemnitz}

\begin{frame}
\frametitle{WISUTEC – Profil}

\begin{itemize}
  \item Zweigniederlassung der G.E.O.S. in Chemnitz
  \item internationale Projekte und digitale Lösungen
  \item Schwerpunkte:
  \begin{itemize}
    \item Bergbauschließung \& Sanierung
    \item Geoinformatik
    \item Strahlenschutz
    \item Umweltmonitoring
  \end{itemize}
\end{itemize}
\end{frame}

% ---------------------------------------------------------
% GEOINFORMATIK
% ---------------------------------------------------------
\section{Fachbereich}
\setoverheader{Fachbereich Geoinformatik}

\begin{frame}
\frametitle{Leistungen}

\begin{columns}[T]  % T = oben ausrichten
  % Linke Spalte: Text
  \begin{column}{0.6\textwidth}
    \begin{itemize}
      \item AL.VIS
      \begin{itemize}
        \item Geologische Messdaten
        \item verortete Geometrien und Objekte
        \item Daten-Monitoring
        \item 3D-Modelle
      \end{itemize}
      \item Installation und Konfiguaration \\der Software- und Serverinfrastruktur
      \item Schulung an den Produkten
    \end{itemize}
  \end{column}

  \hspace{-1.1cm}
  \begin{column}{0.4\textwidth}
    \includegraphics[width=\linewidth,height=0.4\textheight,keepaspectratio]{3d-modell.png}
    \tiny AL.VIS Objects (Closed Source Firmenprodukt)
  \end{column}
\end{columns}

\end{frame}


% ---------------------------------------------------------
% PERSÖNLICHE AUFGABEN
% ---------------------------------------------------------
\section{Eigene Tätigkeit}
\setoverheader{Eigene Themen}

\begin{frame}

\begin{itemize}
  \item Einsatz von KI für interne und externe Prozesse
  \item lokale KI-Chatanwendung für sensible Daten
  \item Bildsuche mithilfe von CLIP
\end{itemize}
\end{frame}

% ---------------------------------------------------------
% ABSCHLUSS
% ---------------------------------------------------------
\setoverheader{Abschluss}

\begin{frame}
\begin{center}
{\Large Vielen Dank für Ihre Aufmerksamkeit!}
\end{center}
\end{frame}

% ---------------------------------------------------------
% Quellen
% ---------------------------------------------------------

\setoverheader{Quellen}
\begin{frame}
  \begin{itemize}
    \item G.E.O.S. Ingenieurgesellschaft mbH, https://www.geosfreiberg.de/, abgerufen am 6. Feb. 2026
    \item WISUTEC, https://www.geosfreiberg.de/de/wisutec.html, abgerufen am 6. Feb. 2026
    \item Eigene Praxiserfahrungen im Rahmen der Tätigkeit bei G.E.O.S./WISUTEC
  \end{itemize}
\end{frame}

\end{document}
