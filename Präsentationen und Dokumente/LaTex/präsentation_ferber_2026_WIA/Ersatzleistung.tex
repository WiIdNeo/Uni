\documentclass[aspectratio=43]{beamer}

% ---------------------------------------------------------
% PAKETE
% ---------------------------------------------------------
\usepackage{xcolor}
\usepackage{graphicx}

% ---------------------------------------------------------
% FARBEN
% ---------------------------------------------------------
\definecolor{mygreen}{RGB}{14,127,38}
\definecolor{lightgraybg}{RGB}{217,217,217}

% ---------------------------------------------------------
% BEAMER-GRUNDLAGEN
% ---------------------------------------------------------
\setbeamertemplate{navigation symbols}{}

\setbeamercolor{headlinebox}{bg=white, fg=black}
\setbeamercolor{footlinebox}{bg=white, fg=black}
\setbeamercolor{frametitle}{fg=mygreen}
\setbeamercolor{normal text}{fg=black}

% ---------------------------------------------------------
% KOPFZEILE (LOGO)
% ---------------------------------------------------------
\setbeamertemplate{headline}{
  \begin{beamercolorbox}[wd=\paperwidth,ht=1.2cm,dp=0cm]{headlinebox}
    \vspace{0.15cm}
    \hfill
    \raisebox{-0.2cm}{\includegraphics[height=1cm]{logo.png}}
    \hspace{0.3cm}

    {\color{mygreen}\rule{\paperwidth}{2pt}}
  \end{beamercolorbox}
}

% ---------------------------------------------------------
% FUSSZEILE (KORRIGIERT)
% ---------------------------------------------------------
\setbeamertemplate{footline}{
  \begin{beamercolorbox}[wd=\paperwidth,ht=0.6cm,dp=0.1cm]{footlinebox}
    \vspace{-0.1cm}
    \hspace{0.3cm}\insertshorttitle
    \hfill\insertframenumber\hspace{0.3cm}
  \end{beamercolorbox}
}

% ---------------------------------------------------------
% HINTERGRUND
% ---------------------------------------------------------
\setbeamertemplate{background}{
  {\color{lightgraybg}\rule{\paperwidth}{\paperheight}}
}

% ---------------------------------------------------------
% TITELINFOS
% ---------------------------------------------------------
\title{Vorstellung: Firma, Bereich und Tätigkeit}
\author{Colin Hanschmann}
\date{\today}

% ---------------------------------------------------------
% DOKUMENT
% ---------------------------------------------------------
\begin{document}

% ---------------------------------------------------------
% ERSTE FOLIE: NUR BILD (OHNE HEADER / FOOTER)
% ---------------------------------------------------------
{
\setbeamertemplate{headline}{}
\setbeamertemplate{footline}{}
\setbeamertemplate{background}{}

\begin{frame}[plain]
  \includegraphics[width=\paperwidth,height=\paperheight]{top_frame.png}
\end{frame}
}

% ---------------------------------------------------------
% TITELFOLIE
% ---------------------------------------------------------
\begin{frame}
  \titlepage
\end{frame}

% ---------------------------------------------------------
% G.E.O.S.
% ---------------------------------------------------------
\begin{frame}
\frametitle{G.E.O.S. – Überblick}

\begin{itemize}
  \item International tätiges Ingenieurunternehmen mit fast 150 Jahren Geschichte
  \item Schwerpunkte:
  \begin{itemize}
    \item Geotechnik
    \item Rohstoffwirtschaft
    \item Umwelt \& Sanierung
    \item Energieinfrastruktur
    \item Hydrogeologie
    \item Geoinformatik
    \item Strahlenschutz
  \end{itemize}
  \item Interdisziplinäre Teams und moderne digitale Methoden
\end{itemize}

\end{frame}

\begin{frame}
\frametitle{G.E.O.S. – Kennzahlen}

\begin{itemize}
  \item 225 Spezialisten
  \item 8 Standorte
  \item Über 1.300 Kunden
  \item Rund 500 Projekte pro Jahr
  \item Kombination aus geowissenschaftlicher Expertise und Ingenieurtechnik
\end{itemize}

\end{frame}

% ---------------------------------------------------------
% WISUTEC
% ---------------------------------------------------------
\begin{frame}
\frametitle{WISUTEC – Profil}

\begin{itemize}
  \item Zweigniederlassung der G.E.O.S. in Chemnitz
  \item Über 45 Jahre Erfahrung aus Uranerzbergbau und Wismut-Sanierung
  \item Internationale Projekte und digitale Lösungen
  \item Schwerpunkte:
  \begin{itemize}
    \item Bergbauschließung \& Sanierung
    \item Geoinformatik
    \item Strahlenschutz
    \item Umweltmonitoring
  \end{itemize}
\end{itemize}

\end{frame}

% ---------------------------------------------------------
% GEOINFORMATIK
% ---------------------------------------------------------
\begin{frame}
\frametitle{Geoinformatik – Aufgaben}

\begin{itemize}
  \item Erfassung, Verarbeitung und Analyse räumlicher Daten
  \item Typische Datenquellen:
  \begin{itemize}
    \item Geologische Messdaten
    \item Monitoring-Daten
    \item Fernerkundung / Satellitendaten
    \item 3D-Modelle
  \end{itemize}
  \item Visualisierung komplexer Zusammenhänge
\end{itemize}

\end{frame}

\begin{frame}
\frametitle{Geoinformatik – Bedeutung für WISUTEC}

\begin{itemize}
  \item Grundlage für Altbergbau- und Sanierungsprojekte
  \item Präzise räumliche Modelle für Risikoanalysen
  \item Unterstützung von Strahlenschutz und Umweltmonitoring
  \item Digitale Planungsgrundlagen
\end{itemize}

\end{frame}

% ---------------------------------------------------------
% ABSCHLUSS
% ---------------------------------------------------------
\begin{frame}
\frametitle{Fragen?}

{\Large Vielen Dank für Ihre Aufmerksamkeit!}

\end{frame}

\end{document}
