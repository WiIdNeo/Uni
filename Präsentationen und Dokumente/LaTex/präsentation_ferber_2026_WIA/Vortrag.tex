\documentclass[aspectratio=43]{beamer}

% Farben
\usepackage{xcolor}
\definecolor{mygreen}{RGB}{14,127,38}
\definecolor{lightgraybg}{RGB}{217,217,217}

% Logo oben rechts
\titlegraphic{\includegraphics[height=1cm]{logo.png}}

% Kopfzeile
\setbeamertemplate{headline}{
    \begin{beamercolorbox}[wd=\paperwidth,ht=1.2cm,dp=0cm]{white}
        \vspace{0.2cm}
        \hfill\raisebox{-0.2cm}{\titlegraphic}\hspace{0.3cm}
        \vspace{0.2cm}
        \color{mygreen}\rule{\paperwidth}{2pt}
    \end{beamercolorbox}
}

% Fußzeile
\setbeamertemplate{footline}{
    \begin{beamercolorbox}[wd=\paperwidth,ht=0.8cm,dp=0cm]{white}
        \hspace{0.3cm}\insertshorttitle
        \hfill\insertframenumber\hspace{0.3cm}
    \end{beamercolorbox}
}

% Überschriftenfarbe
\setbeamercolor{frametitle}{fg=mygreen}

% Inhaltsbereich einfärben
\setbeamercolor{normal text}{bg=lightgraybg, fg=black}

% Hintergrund setzen
\setbeamertemplate{background}{
    \color{lightgraybg}\rule{\paperwidth}{\paperheight}
}

\title{Vorstellung: Firma, Bereich und Tätigkeit}
\author{Colin Hanschmann}
\date{\today}

\begin{document}

% TITELFOLIE
\begin{frame}
    \titlepage
\end{frame}

% ---------------------------------------------------------
% FOLIENBLOCK 1: G.E.O.S.
% ---------------------------------------------------------

\begin{frame}
\frametitle{G.E.O.S. – Überblick}

\begin{itemize}
    \item International tätiges Ingenieurunternehmen mit fast 150 Jahren Geschichte
    \item Schwerpunkte:
    \begin{itemize}
        \item Geotechnik
        \item Rohstoffwirtschaft
        \item Umwelt \& Sanierung
        \item Energieinfrastruktur
        \item Hydrogeologie
        \item Geoinformatik
        \item Strahlenschutz
    \end{itemize}
    \item Interdisziplinäre Teams und moderne digitale Methoden
\end{itemize}

\end{frame}

\begin{frame}
\frametitle{G.E.O.S. – Kennzahlen}

\begin{itemize}
    \item 225 Spezialisten
    \item 8 Standorte
    \item Über 1.300 Kunden
    \item Rund 500 Projekte pro Jahr
    \item Kombination aus geowissenschaftlicher Expertise und Ingenieurtechnik
\end{itemize}

\end{frame}

% ---------------------------------------------------------
% FOLIENBLOCK 2: WISUTEC
% ---------------------------------------------------------

\begin{frame}
\frametitle{WISUTEC – Profil}

\begin{itemize}
    \item Zweigniederlassung der G.E.O.S. in Chemnitz
    \item Über 45 Jahre Erfahrung aus Uranerzbergbau und Wismut-Sanierung
    \item Internationale Projekte und digitale Lösungen
    \item Schwerpunkte:
    \begin{itemize}
        \item Bergbauschließung \& Sanierung
        \item Geoinformatik
        \item Strahlenschutz
        \item Umweltmonitoring
    \end{itemize}
\end{itemize}

\end{frame}

% ---------------------------------------------------------
% FOLIENBLOCK 3: GEOINFORMATIK
% ---------------------------------------------------------

\begin{frame}
\frametitle{Geoinformatik – Aufgaben}

\begin{itemize}
    \item Erfassung, Verarbeitung und Analyse räumlicher Daten
    \item Typische Datenquellen:
    \begin{itemize}
        \item geologische Messdaten
        \item Monitoring-Daten
        \item Fernerkundung / Satellitendaten
        \item 3D-Modelle
    \end{itemize}
    \item Visualisierung komplexer geowissenschaftlicher Zusammenhänge
\end{itemize}

\end{frame}

\begin{frame}
\frametitle{Geoinformatik – Bedeutung für WISUTEC}

\begin{itemize}
    \item Grundlage für Altbergbau- und Sanierungsprojekte
    \item Präzise räumliche Modelle für Risikoanalysen
    \item Unterstützung von Strahlenschutz und Umweltmonitoring
    \item Digitale Planungsgrundlagen für Infrastrukturprojekte
    \item Erwartung der Kunden: verständliche, visuelle Darstellungen
\end{itemize}

\end{frame}

% ---------------------------------------------------------
% ABSCHLUSS
% ---------------------------------------------------------

\begin{frame}
\frametitle{Fragen?}

\Large Vielen Dank für Ihre Aufmerksamkeit!
\end{frame}

\end{document}
