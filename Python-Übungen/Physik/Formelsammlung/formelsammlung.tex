\documentclass[a4paper,12pt]{article}

\usepackage{amsmath, amssymb}
\usepackage{physics}
\usepackage{siunitx}
\usepackage{tcolorbox}
\usepackage{geometry}
\usepackage{setspace}

\geometry{margin=2cm}
\setstretch{1.2}

% Box für Hinweise
\newtcolorbox{note}{
  colback=gray!10,
  colframe=gray!60,
  sharp corners,
  boxrule=0.5pt,
  left=6pt,right=6pt,top=6pt,bottom=6pt
}

\begin{document}

\begin{center}
    {\LARGE \textbf{Formelsammlung – Physik}}\\[4mm]
    Aktualisierte und korrigierte Version
\end{center}

\section{Kreisbewegung mit konstanter Winkelbeschleunigung}

\[
\alpha = \frac{\Delta \omega}{\Delta t}
\]
\[
\omega(t) = \omega_0 + \alpha t
\]
\[
v = \omega r
\]
\[
\Delta\varphi = \omega_0 t + \frac{1}{2}\alpha t^2
\]
\[
\varphi_{\text{end}} = \varphi_{\text{start}} + \Delta\varphi
\]

\begin{note}
Die Angabe $\omega_0 = -\alpha t$ ist nicht allgemein gültig.  
Die Standardform lautet: $\omega(t)=\omega_0+\alpha t$.
\end{note}

% ------------------------------------------------------------

\section{Zwei-Massen-System auf schiefer Ebene}

Kräfte auf Masse $m_1$:
\[
F_g = m_1 g \sin\alpha,\qquad
F_r = \mu m_1 g \cos\alpha
\]

Kräfte auf Masse $m_2$:
\[
F_g = m_2 g
\]

Bewegungsgleichung:
\[
m_2 g - (m_1 g \sin\alpha + \mu m_1 g \cos\alpha)
= (m_1 + m_2)a
\]

Beschleunigung:
\[
a = \frac{m_2 g - m_1 g (\sin\alpha + \mu \cos\alpha)}{m_1 + m_2}
\]

\begin{note}
Gültig bei masselosem, reibungsfreiem Seil sowie masseloser Rolle.  
Haftreibung separat prüfen.
\end{note}

% ------------------------------------------------------------

\section{Feder-Masse-System mit Dämpfung}

\[
k = \frac{F}{\Delta x}
\]
\[
\omega_0 = \sqrt{\frac{k}{m}}
\]
\[
\omega_d = 2\pi f_d
\]

Viskose Dämpfung:
\[
\omega_d = \sqrt{\omega_0^2 - \left(\frac{c}{2m}\right)^2}
\]

\begin{note}
$\delta = \sqrt{\omega_0^2 - \omega_d^2}$ ist korrekt, wenn $\delta = \frac{c}{2m}$.  
Gängige Literatur verwendet das Dämpfungsverhältnis  
\[
\zeta = \frac{c}{2\sqrt{mk}}.
\]
\end{note}

% ------------------------------------------------------------

\section{Proton im elektrischen Feld}

\subsection*{Beschleunigung durch Spannung}
\[
v_x = \sqrt{\frac{2 q U_B}{m}}
\]

\subsection*{Flugzeit im Kondensator}
\[
t = \frac{L}{v_x}
\]

\subsection*{Ablenkung im Feld}
\[
a_y = \frac{qE}{m},\qquad
v_y = a_y t,\qquad
y = \frac{1}{2} a_y t^2
\]

\begin{note}
Gültig im nicht-relativistischen Bereich.  
Bei hohen Spannungen relativistische Effekte berücksichtigen.
\end{note}

% ------------------------------------------------------------

\section{Elektromagnetische Welle}

Zusammenhang der Felder:
\[
H = \frac{E}{Z}
\]

Intensität (RMS-Werte):
\[
I = \frac{E_{\text{eff}}^2}{Z}
\]

\begin{note}
Für Spitzenwerte gilt:
\[
I = \frac{E_{\text{peak}}^2}{2Z}.
\]
Elektromagnetische Wellen sind transversal.
\end{note}

% ------------------------------------------------------------

\section{Dipol und Wellenausbreitung}

Im Vakuum:
\[
\lambda = \frac{c}{f},\qquad
T = \frac{1}{f},\qquad
l = \frac{\lambda}{2}
\]

Im Medium:
\[
v = \frac{c}{\sqrt{\varepsilon_r}}
\]
\[
\lambda' = \frac{v}{f}
\]
\[
l' = \frac{\lambda'}{2}
\]

\begin{note}
Die Frequenz bleibt beim Eintritt in ein Medium unverändert.  
Nur die Wellenlänge ändert sich.
\end{note}

% ------------------------------------------------------------

\begin{center}
\rule{0.8\textwidth}{0.4pt}\\[3pt]
\textit{Ende der Formelsammlung}
\end{center}

\end{document}
